\documentclass[11pt,a4paper]{article}
\usepackage[T1]{fontenc}
\usepackage{amsmath}
\usepackage{amssymb}
\usepackage{graphicx}
\usepackage[UTF8,heading=true]{ctex}
\usepackage{geometry}
\usepackage{diagbox}
\usepackage[]{float}
\usepackage{xeCJK}
\usepackage{indentfirst}
\usepackage{multirow}
\usepackage[section]{placeins}
\usepackage{caption}
\usepackage{cite}
\usepackage{graphics}
\usepackage{subfig}

\graphicspath{{./figure/}}

\setCJKfamilyfont{zhsong}[AutoFakeBold = {5.6}]{STSong}
\newcommand*{\song}{\CJKfamily{zhsong}}

\geometry{a4paper,left=2cm,right=2cm,top=0.75cm,bottom=2.54cm}


\ctexset{
    section={
        format+=\raggedright\song\large
    },
    subsection={
        name={\quad,.}
    },
    subsubsection={
        name={\qquad,.}
    }
}

\begin{document}
\noindent

\begin{center}

    \textbf{\song \zihao{-2} \ziju{0.5}气垫导轨实验预习报告}
    
\end{center}




\section{实验目的及要求}
1. 观察简谐振动现象,测定简谐振动的周期。

2. 求弹簧的倔强系数$k$和有效质量$m_0$。

3. 观察简谐振动的运动学特征。

4. 验证机械能守恒定律。

5. 用极限法测定瞬时速度。

6. 深入了解平均速度和瞬时速度的关系。

\section{实验仪器}
    气垫导轨、滑块、附加砝码、弹簧、U 型挡光片、平板挡光片、数字毫秒计、天平等
\section{实验原理}
\subsection{弹簧振子的间谐运动}
在水平的气垫导轨上,两个相同的弹簧中间系一个滑块,滑块做往返振动,
若不考虑滑块运动的阻力,可以认为滑块的振动是理想的简谐振动。

设质量为$m_1$的滑块初始时处于平衡位置,此时每个弹簧的初始伸长量为$x_0$,当滑块偏离平衡点x
时,受弹性力$-k_1(x+x_0)$与$-k_1(x-x_0)$的作用,其中$k_1$是弹簧的倔强系数。根据牛顿第二定律,列出其运
动方程:$ - kx = m\ddot x$(式中 $k = 2 k_1$)

式中的 $𝑚$与弹簧质量$m_1$并不相同。因为事实上弹簧也是有一定质量的,这导致了实际的运动并非严
格的简谐振动,而是需要考虑弹簧内部形成的驻波,详细推导需要采用分离变量法解微分方程,这里直接
给出结果:若在近似的仍欲采用简谐振动的结论,则可考虑只取一级近似,引入“弹簧有效质量”$m_0$

由一级近似可计算得$m = m_1 + m_0$,$m_0$为弹簧质量的$\frac{1}{3}$,这样对应该方程的解为:
\begin{equation}
    x = A\sin ({\omega _0}t + {\varphi _0})\quad {\omega _0} = \sqrt {\frac{k}{m}} 
\end{equation}

其中周期与固有频率的关系为
\begin{equation}
    T = \frac{{2\pi }}{{{\omega _0}}} = 2\pi \sqrt {\frac{m}{k}}  = 2\pi \sqrt {\frac{{{m_1} + {m_0}}}{k}} 
\end{equation}

将上式两边平方可以得到
\begin{equation}
    {T^2} = \frac{{4{\pi ^2}\left( {{m_1} + {m_0}} \right)}}{k}
\end{equation}

在实验中,我们改变$m_1$,测出相应的𝑇,采用作图法获得$T-m_1$的曲线,理论上该曲线应为一条直
线,直线的斜率为$\frac{4 \pi^2}{k}$,采用最小二乘法可以计算出该直线的斜率,进而算出劲度系数到k的值。同
时,可以从该条直线的截距获取$m_0$的值。也可采用逐差法求解k和$m_0$的值。

\subsection{简谐运动的运动学特征}
运动方程两边同时对时间求导,即可得到
\begin{equation}
    v = \frac{{dx}}{{dt}} = A{\omega _0}\cos \left( {{\omega _0}t + {\varphi _0}} \right)
\end{equation}

由此可见,速度v与时间有关,且随时间的变化关系也为简谐振动,角频率为$\omega_0$,振幅为$A \omega_0$,而且
度v的相位比位移x超前$\frac{\pi}{2}$

联立x-t方程与v-t方程,消去时间t,即可得到
\begin{equation}
    {v^2} = \omega _0^2\left( {{A^2} - {x^2}} \right)
\end{equation}

当x=A时,v=0;当x=0时,$v =  \pm A{\omega _0}$,此时v取最大值

本实验可以通过观察x和v随时间的变化规律,以及x和v之间的相位关系。利用线性拟合的方法算出角频
率

\subsection{简谐振动的机械能}
在实验中,任何时刻系统的振动动能为
\begin{equation}
    {E_k} = \frac{1}{2}m{v^2} = \frac{1}{2}\left( {{m_1} + {m_2}} \right){v^2}
\end{equation}

由于此前在第一个实验项目中,已经测得弹簧的劲度系数为k,因此可以直接算得系统的弹性势能为
(以$m_1$位于平衡位置时系统的势能为零)

\begin{equation}
    {E_p} = \frac{1}{2}k{x^2}
\end{equation}

所以系统的机械能为
\begin{equation}
    E = {E_k} + {E_p} = \frac{1}{2}m{\omega ^2}{A^2} = \frac{1}{2}k{A^2}
\end{equation}

上式中的k和A均不随时间变化

通过测量滑块$m_1$在不同位置x的速度v,从而计算弹性势能和振动势能,并验证他们之间的相互转换
关系和机械能守恒定律是否吻合。

\subsection{瞬时速度的测量}
设变速运动的物体在∆𝑡时间中经过的路程为$\Delta s$,则其平均速度为$\overline v  = \frac{{\Delta s}}{{\Delta t}}$

当$\Delta t$与$\Delta s$均趋于0时,平均速度的极限就为物体的瞬时速度。

在实验中,在倾斜的气轨上,于A点处放置一光电门,在滑块上先后安装上挡光距离不同的U形
挡光片,使各挡光片的第一挡光边距A点为l。滑块每次自P点由静止开始下滑,分别测出相应的挡光
时间$\Delta t$及挡光距离$\Delta s$。(设滑块由静止下滑距离l后的瞬时速度为$v_0$即第一挡光时滑块的瞬时速度),
则有:
\begin{equation}
    \overline v  = \frac{{\Delta s}}{{\Delta t}} = {v_0} + \frac{1}{2}a \cdot \Delta t
\end{equation}

其中a为物体在A附近的加速度
本实验可以通过改变挡光距离$\Delta s$观察平均速度和瞬时速度的关系,分别画出 v-t 图和 v-x 图,利用外
推法求出瞬时速度。
\end{document}