\documentclass[11pt,a4paper]{article}
\usepackage[T1]{fontenc}
\usepackage{amsmath}
\usepackage{amssymb}
\usepackage{graphicx}
\usepackage[UTF8,heading=true]{ctex}
\usepackage{geometry}
\usepackage{diagbox}
\usepackage[]{float}
\usepackage{xeCJK}
\usepackage{indentfirst}
\usepackage{multirow}
\usepackage[section]{placeins}
\usepackage{caption}
\usepackage{cite}
\usepackage{graphics}
\usepackage{subfig}

\graphicspath{{./figure/}}

\setCJKfamilyfont{zhsong}[AutoFakeBold = {5.6}]{STSong}
\newcommand*{\song}{\CJKfamily{zhsong}}

\geometry{a4paper,left=2cm,right=2cm,top=0.75cm,bottom=2.54cm}

\newcommand{\experiName}{气轨上弹簧振子简谐振动及瞬时速度的测定}%实验名称
\newcommand{\supervisor}{纪爱玲}%指导教师
\newcommand{\name}{张钰堃}
\newcommand{\studentNum}{2022k8009926020}
\newcommand{\class}{2}%班级
\newcommand{\group}{08}%组
\newcommand{\seat}{11}%座位号
\newcommand{\dateYear}{2023}
\newcommand{\dateMonth}{10}%月
\newcommand{\dateDay}{17}%日
\newcommand{\room}{教学楼716}%地点
\newcommand{\others}{$\square$}

\ctexset{
    section={
        format+=\raggedright\song\large
    },
    subsection={
        name={\quad,.}
    },
    subsubsection={
        name={\qquad,.}
    }
}

\begin{document}
\noindent

\begin{center}

    \textbf{\song \zihao{-2} \ziju{0.5}《基础物理实验》实验报告}
    
\end{center}


\begin{center}
    \kaishu \zihao{5}
    \noindent \emph{实验名称}\underline{\makebox[28em][c]{\experiName}}
    \emph{指导教师}\underline{\makebox[9em][c]{\supervisor}}\\
    \emph{姓名}\underline{\makebox[6em][c]{\name}} 
    \emph{学号}\underline{\makebox[14em][c]{\studentNum}}
    \emph{分班分组及座号} \underline{\makebox[5em][c]{\class \ -\ \group \ -\ \seat }\emph{号}} (\emph{例}:\,1- 04- 5\emph{号})\\
    \emph{实验日期} \underline{\makebox[3em][c]{\dateYear}} \emph{年}
    \underline{\makebox[2em][c]{\dateMonth}}\emph{月}
    \underline{\makebox[2em][c]{\dateDay}}\emph{日}
    \emph{实验地点}\underline{{\makebox[4em][c]\room}}
    \emph{调课/补课} \underline{\makebox[3em][c]{否}}
    \emph{成绩评定} \underline{\hspace{8em}}
    {\noindent}
    \rule[5pt]{17.7cm}{0.2em}

\end{center}

\section{实验目的及要求}
1. 观察简谐振动现象,测定简谐振动的周期。

2. 求弹簧的倔强系数$k$和有效质量$m_0$。

3. 观察简谐振动的运动学特征。

4. 验证机械能守恒定律。

5. 用极限法测定瞬时速度。

6. 深入了解平均速度和瞬时速度的关系。
\section{实验仪器}
气垫导轨、滑块、附加砝码、弹簧、U 型挡光片、平板挡光片、数字毫秒计、天平等。
\section{实验原理}

    \subsection{弹簧振子的间谐运动}
    在水平的气垫导轨上,两个相同的弹簧中间系一个滑块,滑块做往返振动,
    若不考虑滑块运动的阻力,可以认为滑块的振动是理想的简谐振动。

    设质量为$m_1$的滑块初始时处于平衡位置,此时每个弹簧的初始伸长量为$x_0$,当滑块偏离平衡点x
    时,受弹性力$-k_1(x+x_0)$与$-k_1(x-x_0)$的作用,其中$k_1$是弹簧的倔强系数。根据牛顿第二定律,列出其运
    动方程:$ - kx = m\ddot x$(式中 $k = 2 k_1$)

    式中的 $𝑚$与弹簧质量$m_1$并不相同。因为事实上弹簧也是有一定质量的,这导致了实际的运动并非严
    格的简谐振动,而是需要考虑弹簧内部形成的驻波,详细推导需要采用分离变量法解微分方程,这里直接
    给出结果:若在近似的仍欲采用简谐振动的结论,则可考虑只取一级近似,引入“弹簧有效质量”$m_0$

    由一级近似可计算得$m = m_1 + m_0$,$m_0$为弹簧质量的$\frac{1}{3}$,这样对应该方程的解为:
    \begin{equation}
        x = A\sin ({\omega _0}t + {\varphi _0})\quad {\omega _0} = \sqrt {\frac{k}{m}} 
    \end{equation}

    其中周期与固有频率的关系为
    \begin{equation}
        T = \frac{{2\pi }}{{{\omega _0}}} = 2\pi \sqrt {\frac{m}{k}}  = 2\pi \sqrt {\frac{{{m_1} + {m_0}}}{k}} 
    \end{equation}

    将上式两边平方可以得到
    \begin{equation}
        {T^2} = \frac{{4{\pi ^2}\left( {{m_1} + {m_0}} \right)}}{k}
    \end{equation}

    在实验中,我们改变$m_1$,测出相应的𝑇,采用作图法获得$T-m_1$的曲线,理论上该曲线应为一条直
    线,直线的斜率为$\frac{4 \pi^2}{k}$,采用最小二乘法可以计算出该直线的斜率,进而算出劲度系数到k的值。同
    时,可以从该条直线的截距获取$m_0$的值。也可采用逐差法求解k和$m_0$的值。

    \subsection{简谐运动的运动学特征}
    运动方程两边同时对时间求导,即可得到
    \begin{equation}
        v = \frac{{dx}}{{dt}} = A{\omega _0}\cos \left( {{\omega _0}t + {\varphi _0}} \right)
    \end{equation}

    由此可见,速度v与时间有关,且随时间的变化关系也为简谐振动,角频率为$\omega_0$,振幅为$A \omega_0$,而且
    度v的相位比位移x超前$\frac{\pi}{2}$

    联立x-t方程与v-t方程,消去时间t,即可得到
    \begin{equation}
        {v^2} = \omega _0^2\left( {{A^2} - {x^2}} \right)
    \end{equation}

    当x=A时,v=0;当x=0时,$v =  \pm A{\omega _0}$,此时v取最大值

    本实验可以通过观察x和v随时间的变化规律,以及x和v之间的相位关系。利用线性拟合的方法算出角频
    率

    \subsection{简谐振动的机械能}
    在实验中,任何时刻系统的振动动能为
    \begin{equation}
        {E_k} = \frac{1}{2}m{v^2} = \frac{1}{2}\left( {{m_1} + {m_2}} \right){v^2}
    \end{equation}

    由于此前在第一个实验项目中,已经测得弹簧的劲度系数为k,因此可以直接算得系统的弹性势能为
    (以$m_1$位于平衡位置时系统的势能为零)

    \begin{equation}
        {E_p} = \frac{1}{2}k{x^2}
    \end{equation}

    所以系统的机械能为
    \begin{equation}
        E = {E_k} + {E_p} = \frac{1}{2}m{\omega ^2}{A^2} = \frac{1}{2}k{A^2}
    \end{equation}

    上式中的k和A均不随时间变化

    通过测量滑块$m_1$在不同位置x的速度v,从而计算弹性势能和振动势能,并验证他们之间的相互转换
    关系和机械能守恒定律是否吻合。

    \subsection{瞬时速度的测量}
    设变速运动的物体在∆𝑡时间中经过的路程为$\Delta s$,则其平均速度为$\overline v  = \frac{{\Delta s}}{{\Delta t}}$

    当$\Delta t$与$\Delta s$均趋于0时,平均速度的极限就为物体的瞬时速度。

    在实验中,在倾斜的气轨上,于A点处放置一光电门,在滑块上先后安装上挡光距离不同的U形
    挡光片,使各挡光片的第一挡光边距A点为l。滑块每次自P点由静止开始下滑,分别测出相应的挡光
    时间$\Delta t$及挡光距离$\Delta s$。(设滑块由静止下滑距离l后的瞬时速度为$v_0$即第一挡光时滑块的瞬时速度),
    则有:
    \begin{equation}
        \overline v  = \frac{{\Delta s}}{{\Delta t}} = {v_0} + \frac{1}{2}a \cdot \Delta t
    \end{equation}

    其中a为物体在A附近的加速度
    本实验可以通过改变挡光距离$\Delta s$观察平均速度和瞬时速度的关系,分别画出 v-t 图和 v-x 图,利用外
    推法求出瞬时速度。

\section{实验内容}
    
1. 学会光电门测速和测周期的使用方法。

2. 调节气垫导轨至水平状态,通过测量任意两点的速度变化,验证气垫导轨是否处于
水平状态。

3. 测量弹簧振子的振动周期并考察振动周期和振幅的关系。滑块的振幅 A 分别取
10.0、 20.0、 30.0、 40.0 cm 时,测量其相应振动周期。分析和讨论实验结果可得出什么
结论? 

4. 研究振动周期和振子质量之间的关系。在滑块上加骑码(铁片)。对一个确定的振幅
(如取 A=40.0 cm)每增加一个骑码测量一组 𝑇。(骑码不能加太多,以阻尼不明显为限。) 
作 $T^2-m$ 的图,如果 $T$与 $m$ 的关系式如公式(6)所示,则$T^2-m$ 的图应为一条直线,
其斜率为 $4$$\pi^2$/$k$,截距为$4$$\pi^2$$m_0$/$k$ 。用最小二乘法做直线拟合,求出 $k$ 和 $m_0$。

5. 研究速度和位移的关系。在滑块上装上 U 型挡光片,可测量速度。
作 $v^2-x^2$的图,看该图是否为一条直线,并进行直线拟合,看斜率是否为 $-\omega_0^2$ ,截
距是否为 $A^2$$\omega_0^2$,其中$\omega_0=2\pi /T $ ,𝑇可测出。

6. 研究振动系统的机械能是否守恒。固定振幅(如取 A=40.0cm),测出不同x处的滑
块速度,由此算出振动过程中经过每一个x处的动能和势能,并对各x处的机械能进行比
较,得出结论。

7. 研究平均速度与瞬时速度的关系,利用外推法求出瞬时速度。在气轨下面只有一个螺丝的一端,小心将气轨抬起来,把垫块放到这个螺丝的下面。测量具有不同$\Delta s$的挡光片在
距离 A 点为 50cm 处从静止开始自由下滑,从 A 点开始在$\Delta s$所用的时间$\Delta t$,求出平均速度$\overline{v}$ ,作$\overline{v}-\Delta t$图和$\overline{v}-\Delta s $图,将图线性外推求出瞬时速度$v_0$。

8. 通过改变气轨的倾斜角度$\theta $(增加垫块数量),重复上述实验。

9. 通过改变 A 点到 P 点的距离𝑙(设置 60cm 处),重复上述实验。
\section{实验数据及数据处理}
    \subsection{实验仪器调试}
    \begin{table}[H]
        \centering
        \begin{tabular}{|l|l|l|l|l|l|l|}
        \hline
            $v_1$(cm/s) & $v_2$(cm/s) & 误差(\%) \\ \hline
            5.71 & 5.69 & 0.35 \\ \hline
            6.05 & 6.03 & 0.33 \\ \hline
            4.78 & 4.76 & 0.47 \\ \hline
        \end{tabular}
    \end{table}
    通过调试使三次误差均低于$0.5\%$,说明导轨已经十分接近水平状态。
    \subsection{测量弹簧振子的振动周期并考察振动周期和振幅的关系}
    滑块的振幅A分别取10.0,20.0,30.0,40.0cm时,测量其振动周期
    \begin{table}[H]
        \centering
        \begin{tabular}{|l|l|l|l|l|l|l|}
        \hline
             & 10cm & 20cm &30cm&40cm \\ \hline
            $T_1$(ms) & 1621.10 & 1620.15 &1620.01&1618.95\\ \hline
            $T_2$(ms) & 1621.42 & 1620.44 &1619.69&1618.99\\ \hline
            $T_3$(ms) & 1621.19 & 1620.10 &1619.63&1619.29\\ \hline
            $T_4$(ms) &1621.37  &1619.96  &1619.63&1619.47\\ \hline
            $T_5$(ms) &1621.34  &1619.97  &1619.87&1619.11\\ \hline
            $T$(ms)   &1621.28  &1620.12  &1619.77&1619.16\\ \hline
        \end{tabular}
    \end{table}
    已知理论上,周期与振幅无关。观察可以发现,当振幅不同时,测得的四个周期值均较为接近,根据实验结果来看可以认为,在误差的允许范围内,周期的大小与振幅无关。
    将4个周期做平均,可以得到带有条形挡光片的滑块做简谐运动的周期大约为1620ms。
    \subsection{研究弹簧振子振动周期与振子质量之间的关系}
    振子的振幅A取40.0cm
    \begin{table}[H]
        \centering
        \begin{tabular}{|l|l|l|l|l|l|l|}
        \hline
            m(g) & 219.99 & 242.20 & 232.40 & 240.98 & 257.39  \\ \hline
            $T_1$  & 1620.68 &1710.95&1666.91&1708.49&1750.68 \\ \hline
            $T_2$  & 1620.58 &1711.18&1666.99&1708.62&1750.89  \\ \hline
            $T_3$  & 1620.65 &1710.95&1667.05&1708.68&1750.65  \\ \hline
            $T_4$  & 1620.52 &1711.16&1667.09&1708.53&1750.73 \\ \hline
            $T_5$  &1620.53&1711.12&1667.09&1708.50&1750.72 \\ \hline
            $T_6$  &1620.45&1711.09&1667.16&1708.18&1750.69 \\ \hline
            $T_7$  &1620.49&1711.06&1667.10&1708.35&1750.59 \\ \hline
            $T_8$  &1620.49&1711.13&1666.69&1708.54&1750.36\\ \hline
            $T_9$  &1620.53&1711.15&1666.88&1708.38&1750.72 \\ \hline
            $T_{10}$ &1620.56&1711.18&1666.86&1708.41&1750.77   \\ \hline
            $T$    &1620.55&1711&1666.99&1708.47&1750.68 \\ \hline

        \end{tabular}
    \end{table}
    绘制图像
    \begin{figure}[H]
        \centering
        \includegraphics[scale=0.71]{图片2.png}
        \caption{振子周期与质量的关系}
    \end{figure}
    根据图像拟合,可知直线斜率为12.01,$R^2$=0.9822 ,十分接近1,说明拟合程度较好。由实验原理部分,可知斜率为$\frac{4{\pi ^2}}{k} $ ,截距为$\frac{4{\pi ^2}m_0}{k}$ ,计算可知,该弹簧的弹性系数为3.29N/m ,弹簧的有效质量为0.22g。
    \subsection{研究速度与位移的关系}
    振子的振幅A取40.0cm
    \begin{table}[H]
        \centering
        \begin{tabular}{|l|l|l|l|l|l|l|}
        \hline
           &10cm&15cm&20cm&25cm&30cm\\\hline
           $v_1$(cm/s)&129.70&119.62&118.20&110.99&90.99\\\hline
           $v_2$(cm/s)&128.53&118.34&116.69&110.01&88.73\\\hline
           $v_3$(cm/s)&126.42&121.95&114.68&107.18&85.62\\\hline
           $v$(cm/s)&128.22&119.97&116.52&109.39&88.45\\\hline
        \end{tabular}
    \end{table}
    绘制图像
    \begin{figure}[H]
        \centering
        \includegraphics[scale=0.71]{图片5.png}
        \caption{速度与位移的关系}
    \end{figure}
    由图像可知,拟合直线斜率为-0.0969,截距为0.1695,$R^2$=0.9621,可见拟合程度较高,由原理部分公式可知,${\omega _0}$=0.311$s^{-1}$
    \subsection{研究机械能是否守恒}
    振子的振幅A取40.0cm
    \begin{table}[H]
        \centering
        \begin{tabular}{|l|l|l|l|l|l|l|}
        \hline
        &10cm&15cm&20cm&25cm&30cm\\\hline
        $v$(cm/s)&128.22&119.97&116.52&109.39&88.45\\\hline
        $E_k$(J)&0.18&0.16&0.15&0.13&0.09\\\hline
        $E_p$(J)&0.017&0.037&0.066&0.104&0.149\\\hline
        $E$(J)&0.20&0.20&0.21&0.21&0.23\\\hline
        \end{tabular}
    \end{table}
    由机械能数据可知,振动系统的机械能在振动过程中大致不变,与理论结果相符
    \subsection{改变振幅A,测出相应的$v_{max}$,由${v_{max}}^2$-$A^2$图像求k}
    \begin{table}[H]
        \centering
        \begin{tabular}{|l|l|l|l|l|l|l|}
            \hline
            &10cm&15cm&20cm&25cm&30cm\\\hline
            $v_{{max}_1}$(cm/s)&47.82&67.61&86.36&107.64&124.07\\\hline
            $v_{{max}_2}$(cm/s)&47.73&67.43&85.91&107.18&123.46\\\hline
            $v_{{max}_3}$(cm/s)&46.51&66.14&84.53&105.60&121.65\\\hline
            $v_{max}$(cm/s)&47.35&67.06&85.60&106.81&123.06\\\hline
          
        \end{tabular}
    \end{table}
    绘制图像
    \begin{figure}[H]
        \centering
        \includegraphics[scale=0.71]{图片4.png}
        \caption{振子最大速度与振幅的关系}
    \end{figure}
    由图像可知,拟合直线斜率为16.264,截距为0.0805,由公式计算可知,k=3.54N/m
    \subsection{其他相关参数}
    滑块的质量:217.34g

    条形挡光片质量:2.63g
    
    U型挡光片质量:11.76g
    \subsection{测定瞬时速度与不同U型挡光片通过光电门所用的时间(AP=50cm),计算平均速度}
    \begin{table}[H]
        \centering
        \begin{tabular}{|l|l|l|l|l|l|l|1|}
            \hline
            &$\Delta t_1$(ms) &$\Delta t_2$(ms)&$\Delta t_3$(ms) &$\Delta t_4$(ms)&$\Delta t_5$ (ms)&$\Delta t$(ms)&$\overline{v}$(m/s)\\\hline
            1cm&60.02&61.82&61.87&61.76&60.70&61.23&0.16 \\\hline
            3cm&196.20&188.02&192.77&188.74&198.06&192.76&0.16\\\hline
            5cm&305.27&313.04&305.77&305.80&300.66&306.11&0.16\\\hline
            10cm&570.02&580.59&578.75&583.19&589.73&580.46&0.17\\\hline
        \end{tabular}
    \end{table}
    由数据可知,随着挡光片的宽度变化,平均速度近似不变,与理论结果相同
    \subsection{改变导轨倾角,测定瞬时速度与不同U型挡光片通过光电门所用的时间(AP=50cm),计算平均速度}
    \begin{table}[H]
        \centering
        \begin{tabular}{|l|l|l|l|l|l|l|1|}
            \hline
            &$\Delta t_1$(ms) &$\Delta t_2$(ms)&$\Delta t_3$(ms) &$\Delta t_4$(ms)&$\Delta t_5$ (ms)&$\Delta t$(ms)&$\overline{v}$(m/s)\\\hline
            1cm&36.86&36.80&36.97&37.19&37.28&37.02&0.27 \\\hline
            3cm&104.36&104.80&103.14&104.03&101.85&103.63&0.29\\\hline
            5cm&184.61&187.02&187.00&182.11&185.55&185.24&0.27\\\hline
            10cm&344.91&344.43&343.21&342.30&340.52&343.07&0.29\\\hline
        \end{tabular}
    \end{table}
    由数据可知,随着导轨倾角变化,平均速度近似不变,与理论结果相同
    \subsection{测定瞬时速度与不同U型挡光片通过光电门所用的时间(AP=60cm),计算平均速度}
    \begin{table}[H]
        \centering
        \begin{tabular}{|l|l|l|l|l|l|l|1|}
            \hline
            &$\Delta t_1$(ms) &$\Delta t_2$(ms)&$\Delta t_3$(ms) &$\Delta t_4$(ms)&$\Delta t_5$ (ms)&$\Delta t$(ms)&$\overline{v}$(m/s)\\\hline
            1cm&33.62&33.95&34.11&34.02&34.11&33.96&0.29 \\\hline
            3cm&97.77&96.96&97.84&96.98&96.95&97.30&0.31\\\hline
            5cm&168.81&169.08&170.89&169.02&170.64&169.69&0.29\\\hline
            10cm&321.67&326.44&319.86&320.28&321.96&322.04&0.31\\\hline
        \end{tabular}
    \end{table}
    由数据可知,随着挡光片的宽度变化,平均速度近似不变,与理论结果相同
    \subsection{总结}
    可以从数据中得知本实验中存在一些误差,比如第三部分和第六部分所算出的k值差别较大,我认为误差可能来自于以下原因:

    1.注意到本次实验中不同挡光片对应的平均速度相差很小,数量级在$10^{-3}$,但是仪器测量精度只能
    达到$10^{-2}$,因此读数的误差对拟合结果的影响较大

    2.在实际实验中很难保证滑块释放时没有初速度,而误差对数据的波动比较敏感

    3.除此之外,实验中存在气垫导轨的摩擦和空气阻力,而在处理中均忽略了这两个因素的影响



\section{反思}
    (1)实验前务必要预习,尤其是透彻理解实验理论与原理:

    如果只是照着书本上的操作步骤,那么实验本身就会变成很枯燥的体验,少了实验过程中探索未
    知,检验理论的体验感,同时也少了很多乐趣。

    实际做实验时几乎可以说必然会遇到各种棘手问题,若不理解理论则很难做到灵活应对这些问题。

    (2)灵活使用计算机软件是处理实验数据必不可少的技能:

    这次实验的绘图我是用 Excel 完成的,其优势是生成曲线迅速,不用代码就能对数据进行各种操
    作,但其精度远远比不上专业的数据处理软件,我还需要进一步的学习。

    (3)认真对待误差分析:

    做实验时,要格外留意会存在哪些带来误差的地方,并且反思这样的误差是否是可以采用别的方
    法,从而尽可能的减少影响。尤其是本次实验测定瞬时速度的部分,误差分析让我受益匪浅。

\section*{附:原始实验数据}
    \begin{figure}[H]
        \centering
        \includegraphics[scale=0.15]{1.jpg}
        \includegraphics[scale=0.15]{2.jpg}
        \includegraphics[scale=0.15]{3.jpg}
    \end{figure}

\end{document}
