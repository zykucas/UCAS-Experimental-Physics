\documentclass[11pt,a4paper]{article}
\usepackage[T1]{fontenc}
\usepackage{amsmath}
\usepackage{amssymb}
\usepackage{graphicx}
\usepackage[UTF8,heading=true]{ctex}
\usepackage{geometry}
\usepackage{diagbox}
\usepackage[]{float}
\usepackage{xeCJK}
\usepackage{indentfirst}
\usepackage{multirow}
\usepackage[section]{placeins}
\usepackage{caption}
\usepackage{cite}
\usepackage{graphics}
\usepackage{subfig}

\graphicspath{{./figure/}}

\setCJKfamilyfont{zhsong}[AutoFakeBold = {5.6}]{STSong}
\newcommand*{\song}{\CJKfamily{zhsong}}

\geometry{a4paper,left=2cm,right=2cm,top=0.75cm,bottom=2.54cm}

\newcommand{\experiName}{RLC电路的谐振与暂态过程}%实验名称
\newcommand{\supervisor}{黄金浩}%指导教师
\newcommand{\name}{张钰堃}
\newcommand{\studentNum}{2022k8009926020}
\newcommand{\class}{2}%班级
\newcommand{\group}{08}%组
\newcommand{\seat}{11}%座位号
\newcommand{\dateYear}{2023}
\newcommand{\dateMonth}{10}%月
\newcommand{\dateDay}{24}%日
\newcommand{\room}{教学楼709}%地点
\newcommand{\others}{$\square$}

\ctexset{
    section={
        format+=\raggedright\song\large
    },
    subsection={
        name={\quad,.}
    },
    subsubsection={
        name={\qquad,.}
    }
}

\begin{document}
\noindent

\begin{center}

    \textbf{\song \zihao{-2} \ziju{0.5}《基础物理实验》实验报告}
    
\end{center}


\begin{center}
    \kaishu \zihao{5}
    \noindent \emph{实验名称}\underline{\makebox[28em][c]{\experiName}}
    \emph{指导教师}\underline{\makebox[9em][c]{\supervisor}}\\
    \emph{姓名}\underline{\makebox[6em][c]{\name}} 
    \emph{学号}\underline{\makebox[14em][c]{\studentNum}}
    \emph{分班分组及座号} \underline{\makebox[5em][c]{\class \ -\ \group \ -\ \seat }\emph{号}} (\emph{例}:\,1- 04- 5\emph{号})\\
    \emph{实验日期} \underline{\makebox[3em][c]{\dateYear}} \emph{年}
    \underline{\makebox[2em][c]{\dateMonth}}\emph{月}
    \underline{\makebox[2em][c]{\dateDay}}\emph{日}
    \emph{实验地点}\underline{{\makebox[4em][c]\room}}
    \emph{调课/补课} \underline{\makebox[3em][c]{否}}
    \emph{成绩评定} \underline{\hspace{8em}}
    {\noindent}
    \rule[5pt]{17.7cm}{0.2em}
\end{center}

\section{实验目的}
1. 研究 RLC 电路的谐振现象。

2. 了解 RLC 电路的相频特性和幅频特性。

3. 用数字存储示波器观察 RLC 串联电路的暂态过程,理解阻尼振动规律。
\section{实验仪器与用具}
标准电感,标准电容,100$\Omega $,标准电阻,电阻箱,电感箱,电容箱,函数发生器,示波器,
数字多用表,导线等。
\section{实验原理}
\subsection{串联谐振}
RLC 串联电路如图 1 所示。其总阻抗$\left\lvert Z\right\rvert $、电压u与电流i之间的相位差$\varphi $、电流i分别为
\begin{equation}
    \left\lvert Z\right\rvert=\sqrt{R^2+(\omega L-\frac{1}{\omega C} )^2}  
\end{equation}
\begin{equation}
    \varphi =\arctan \frac{\omega L-\frac{1}{\omega C} }{R} 
\end{equation}
\begin{equation}
   i= \frac{u}{\sqrt{R^2+(\omega L-\frac{1}{\omega C} )} } 
\end{equation}
式中$\omega =2\pi f$为角频率,$\left\lvert Z\right\rvert $、$\varphi$ 、i都是f的函数
\begin{figure}[H]
    \centering
    \includegraphics[scale=0.7]{1.png}
    \caption{RLC串联电路}
\end{figure}
图 2(a)、(b)、(c)分别为 RLC 串联电路的阻抗、相位差、电流随频率的变化曲线。其中图 2(b)$\varphi -f$曲线称为相频特性曲线;图 2(c)
$i-f$曲线称为幅频特性曲线,它表示在总电压u保持不变的条件下i随f的变化曲线。相频特性曲线和幅频特性曲线有时统称为频率响应特
性曲线。

由曲线图可以看出,存在一个特殊的频率$f_0$,特点为:
(1)当$f<f_0$时,$\varphi <0$,电流的相位超前于电压,整个电路呈电容性,且随f降低,$\varphi$ 趋近于-$\frac{\pi }{2}$;而当$f>f_0$时,$\varphi >0$,电流的相位落后于电压,整个电路呈电感性,且随
f升高,$\varphi$ 趋近于$\frac{\pi }{2}$
(2)随f偏离$f_0$越远,阻抗越大,而电流越小。
(3)当$\omega L-\frac{1}{\omega C}$,即
\begin{equation}
    \omega _0=\frac{1}{\sqrt{LC} } 或 f_0=\frac{1}{2\pi \sqrt{LC} }
\end{equation}
$\varphi  = 0$,电压与电流同相位,整个电路呈纯电阻性,总阻抗达到极小值$Z_0=R$,而总电流达到极大值$i_m=\frac{u}{R}$。这种特殊的状态称为串联谐振,此时角频率
$\omega _0$称为谐振角频率。在$f_0$处,$i − f$曲线有明显尖锐的峰显示其谐振状态,因此,有时称它为谐振曲线。谐振时,有
\begin{equation}
    u_L=i_m\left\lvert Z_L\right\rvert =\frac{{\omega _0}Lu}{R},\frac{u_L}{u}=\frac{{\omega _0}L}{R}=\frac{1}{R}*\sqrt{\frac{L}{C}}
\end{equation}
而
\begin{equation}
    u_C=i_m\left\lvert Z_C\right\rvert =\frac{u}{RC{\omega _0}},\frac{u_C}{u}=\frac{1}{RC{\omega _0}}=\frac{1}{R}*\sqrt{\frac{L}{C}}
\end{equation}
令
\begin{equation}
    Q=\frac{u_C}{u}=\frac{u_L}{u}=\frac{{\omega _0}L}{R}=\frac{1}{RC{\omega _0}}
\end{equation}
Q称为谐振电路的品质因数,简称Q值。它是由电路的固有特性决定的,是标志和衡量谐振电路性能优劣的重要的参量。
\subsection{并联谐振}
如图 3 所示电路,其总阻抗$\left\lvert Z_P\right\rvert$、电压u与电流i之间的相位差$\varphi $、电压u(或电流i)分别为
\begin{equation}
    \left\lvert Z_P\right\rvert=\sqrt{\frac{R^2+(\omega L)^2}{(1-{\omega }^2LC)^2}+(\omega CR)^2} 
\end{equation}
\begin{equation}
    \arctan {\frac{\omega L-\omega C[R^2+(\omega L)^2]}{R}}    
\end{equation}
\begin{equation}
    u-i\left\lvert Z_P\right\rvert =\frac{u_R}{u}\left\lvert Z_P\right\rvert 
\end{equation}
显然,它们都是频率的函数。当$\varphi $=0 时,电流和电压同相位,整个电路呈纯电阻性,即发生谐振。由式(8)求得并联谐振的角频率$\omega _p$
为
\begin{equation}
    \omega _p=2\pi f_p=\sqrt{\frac{1}{LC}-(\frac{R}{L})^2} =\omega _0\sqrt{1-\frac{1}{Q^2}} 
\end{equation}
式中$\omega _0=2\pi f_0=\frac{1}{LC},Q=\frac{\omega _0L}{R}=\frac{1}{R}\sqrt{\frac{L}{C}}$,可见,并联谐振频率$f_p$与$f_0$稍有不同,
当Q >> 1时,$\omega _p\approx \omega _0$, $f_P\approx f_0$。

图 4(a)、(b)、(c)分别为 RLC 并联电路的阻抗、相位差、电流或电压随频率的变化曲线。
由(b)图$\varphi  − f$曲线可见,在谐振频率$f=f_P$两侧,当$f<f_P$,$\varphi >0$,电流的相位落后于电压,整个电路呈电感性;当
$f>f_P$,$\varphi <0$,电流的相位超前于电压,整个电路呈电容性。

显然,在谐振频率两边区域,并联电路的电抗特性与串联电路时截然相反。由(a)图$\left\lvert Z_p\right\rvert-f$线和(c)图i − f
曲线可见,在$f={f_p}^'$处(注意:${f_p}^'$与$f_p$稍有不同)总阻抗达到极大值,总电流达到极小值,而在${f_p}^'$两侧,随f偏离
${f_p}^'$越远,阻抗越小,电流越大。不言而喻,这种特性,与串联电路时完全相反。(c)图u − f曲线为在总电流保持不变的条件下,电感(或电容)两端电压
u随频率的变化曲线。
\begin{figure}[H]
    \centering
    \includegraphics[scale=0.7]{4.png}
\end{figure}
与串联谐振类似,可用品质因数Q,即
\begin{equation}
    Q_1=\frac{\omega _0L}{R}=\frac{1}{R\omega _0C};Q_2=\frac{i_c}{i}\approx \frac{i_L}{i};Q_3=\frac{f_0}{\Delta f}
\end{equation}
标志并联谐振电路的性能优劣,其意义也类同。不过,此时$i_L\approx i_C = iQ$,谐振支路中的电流为总电流的Q倍。因此,有时称并联谐振为电流谐振。
\subsection{ RLC 电路的暂态过程}
电路如图 5。先观察放电过程,即开关 S 先合向“1”使电容充电至 E,然后把 S 倒向“2”,电容就在闭合的
RLC 电路中放电。电路方程为
\begin{figure}[H]
    \centering
    \includegraphics[scale=0.7]{5.png}
\end{figure}
\begin{figure}[H]
    \centering
    \includegraphics[scale=0.7]{19.png}
\end{figure}
又将$ i = C\frac{d{u_C}}{dt}$代入得
\begin{figure}[H]
    \centering
    \includegraphics[scale=0.7]{8.png}
\end{figure}
根据初始条件t = 0,$u_c = E$,$\frac{d{u_c}}{dt}=0$解方程。方程的解分为 3 种情况:

(1)$R^2<\frac{4L}{C}$属于阻尼较小的情况。引入阻尼系数$\zeta  =\frac{R}{2}\sqrt{\frac{C}{L}} $𝐿
后,对应于$\zeta $< 1。此时方程的解为
\begin{figure}[H]
    \centering
    \includegraphics[scale=0.8]{7.png}
\end{figure}
其中时间常量为
\begin{figure}[H]
    \centering
    \includegraphics[scale=0.8]{9.png}
\end{figure}
衰减振动的角频率为
\begin{figure}[H]
    \centering
    \includegraphics[scale=0.8]{10.png}
\end{figure}
$U_C$随时间变化的规律如图曲线 I 所示,即阻尼振动状态。此时振动的振幅呈指数衰减。$\tau $的
大小决定了振幅衰减的快慢,$\tau $越小,振幅衰减越迅速。
\begin{figure}[H]
    \centering
    \includegraphics[scale=0.6]{11.png}
\end{figure}
如果$R^2<<\frac{4L}{C}$,通常是 R 很小的情况,振幅的衰减很缓慢
\begin{figure}[H]
    \centering
    \includegraphics[scale=0.8]{12.png}
\end{figure}
此时近似为 LC 电路地自由振动,$\omega _0$为 R=0 时 LC 回路的固有频率。衰减振动的周期
\begin{figure}[H]
    \centering
    \includegraphics[scale=0.8]{13.png}
\end{figure}
(2)$R^2>\frac{4L}{C}$,即阻尼系数$\zeta $> 1,1。对应于过阻尼状态,其解为
\begin{figure}[H]
    \centering
    \includegraphics[scale=0.8]{14.png}
    \includegraphics[scale=0.8]{15.png}
\end{figure}
式所表示的$u_c$ − 𝑡的关系曲线见图 6 中的曲线 II,它是以缓慢的方式逐渐回零。可以证明,
若 L 和 C 固定,随电阻 R 的增长,$u_c$衰减到零的过程更加缓慢。
(3)$R^2=\frac{4L}{C}$,即阻尼系数$\zeta $= 1。对应于临界阻尼状态,其解为
\begin{figure}[H]
    \centering
    \includegraphics[scale=0.8]{16.png}
\end{figure}
其中$\tau $ = 2𝐿/𝑅。它是从过阻尼到阻尼振动过渡的分界点,$u_c$ − 𝑡的关系见图 6 中的曲线 III。
对于充电过程,即开关 S 先在位置“2”,待电容放电完毕,再把 S 倒向“1”,电源 E 将对电
容充电,于是电路方程变为
\begin{figure}[H]
    \centering
    \includegraphics[scale=0.8]{17.png}
\end{figure}
初始条件为t = 0 时,$u_c$ = 0,$\frac{d{u_c}}{dt} = 0$。方程解为
\begin{figure}[H]
    \centering
    \includegraphics[scale=0.8]{18.png}
\end{figure}
\section{实验内容}
    \subsection{测RLC串联电路的相频特性和幅频特性曲线}
    测量电路参考讲义。调节$L=0.1H,C=0.05\mu F,R=100\Omega$。用示波器CH1、CH2通道分别测量RLC串联电路的总电压$u$和电阻两端电压$u_R$

    改变函数发生器的输出频率,找到谐振频率$f_0$,在谐振时,用数字多用表测量$u$,$u_L$,
    $u_C$。利用$Q=\frac{u_L}{u}=\frac{u_C}{u}$计算Q值

    保持总电压$u_pp=2.0V$不变,用示波器测量电压、电流间相位差$\varphi$以及相应的$u_R$。
    选择相位差约$0^\circ {\rm{,}} \pm 15^\circ {\rm{,}} \pm 30^\circ {\rm{,}} \pm 45^\circ {\rm{,}} \pm 60^\circ {\rm{,}} \pm 72^\circ {\rm{,}} \pm 80^\circ $
    所对应的频率进行测量。参考频率(单位kHz):1.88、2.00、2.08、2.15、2.19、2.22、2.24、2.25、2.26、2.275、
    2.30、2.36、2.43、2.62、3.18。作RLC串联电路的$\varphi-f$曲线和$i-f$曲线。利用
    $Q=\frac{f_0}{\Delta f}$估算出Q值。

    \subsection{测量RLC并联电路的相频特性与幅频特性曲线}
    测量电路参考讲义。取$L=0.1H,C=0.05\mu F,R'=5k\Omega$,用CH1测量总电压,用CH2测量
    $R'$两端电压,两通道测量电压相减CH1-CH2就是并联部分电压$u$

    改变函数发生器的输出频率,观测并联部分的电压$u(CH1-CH2)$与总电流(CH2)的幅度和相位
    的变化,找到谐振频率$f_p$

    测相频特性曲线和幅频特性曲线:固定总电压$(u+u_{R'})$的峰值2.0V保持不变,测量并联
    部分电压$u(CH1-CH2)$与总电流(CH2)的相位差以及二者的幅度值。参考频率(单位kHz):
    2.05、2.15、2.20、2.231、2.24、2.247、2.25、2.253、2.256、2.265、2.275、2.32、2.40、2.60。
    作RLC并联电路的$\varphi-f$曲线和$u-f、i-f$曲线

    \subsection{测量RLC串联电路的暂态过程}
    实验电路参考讲义。调节$L=0.1H,C=0.2\mu F$

    由函数发生器产生方波,函数发生器各参数可设为:频率50Hz,电压峰峰值$u_pp=2.0V$,偏移
    1V。示波器CH1通道用来测量总电压,CH2用来测量电容两端电压$u_C$,注意两个通道必须共地

    调节$R=0\Omega$,测量$u_C$波形

    调节R测得临界电阻,并与理论值比较
    
    分别记录$R=2k\Omega,20k\Omega$的波形。函数发生器频率可分别选为$250Hz(R=2k\Omega)$和
    $20Hz(R=20k\Omega)$

\section{实验数据与处理}
    \subsection{测量RLC串联电路的相频特性和幅频特性曲线}
    \subsubsection{测量谐振频率和品质因子}
    直接测谐振,找到谐振频率为:$f_0=2.2518(kHz)$\par
    用万用表测量得到:$u=0.466(V),u_L=5.56(V),u_C=5.56V$\par
    分别用$u_L$和$u_C$计算品质因子Q,分别得到$Q=11.93$
    \subsubsection{测量相频特性曲线和幅频特性曲线}
    \begin{table}[H]
        \centering
        \begin{tabular}{|l|l|l|l|l|}
        \hline
            f/kHz & U/V & $(CH1-CH2) \varphi/^{\circ}$ & $U_R/V$ & I/mA \\ \hline
            1.88 & 2 & -79.50 & 0.3643 & 3.643 \\ \hline
            2 & 2 & -71.80 & 0.5166 & 5.166 \\ \hline
            2.08 & 2 & -61.30 & 0.697 & 6.97 \\ \hline
            2.15 & 2 & -47.80 & 0.906 & 9.06 \\ \hline
            2.19 & 2 & -32.6 & 1.03 & 10.3 \\ \hline
            2.22 & 2 & -18.4 & 1.098 & 10.98 \\ \hline
            2.24 & 2 & -4.01 & 1.122 & 11.22 \\ \hline
            2.25 & 2 & 2.33 & 1.125 & 11.25 \\ \hline
            2.26 & 2 & 3.3 & 1.124 & 11.24 \\ \hline
            2.275 & 2 & 12.40& 1.11 & 11.1 \\ \hline
            2.3 & 2 & 24.2 & 1.063& 10.63 \\ \hline
            2.36 & 2 & 46 & 0.898 & 8.98 \\ \hline
            2.43 & 2 & 59 & 0.709 & 7.09 \\ \hline
            2.62 & 2 & 72 & 0.405 & 4.05 \\ \hline
            3.18 & 2 & 75 & 0.169 & 1.69 \\ \hline
        \end{tabular}
    \end{table}

    绘制图像:
    \begin{figure}[H]
        \centering
        \includegraphics[scale=0.7]{图片1.png}
    \end{figure}
    \begin{figure}[H]
        \centering
        \includegraphics[scale=0.7]{图片2.png}
    \end{figure}

    \subsubsection{讨论}
        我们从绘制的相频特性曲线可以看出,相位差随着频率的变化由负到正,在谐振频率附近变化
    率较大,而在接近$\pm90^{\circ}$时,变化率几乎为零,逐渐趋于平行 x 轴的直线。在实验误差允许范围内,与
    理论计算所得的图像一致。

        幅频特性曲线中,电流呈先增大后减小的趋势,在谐振频率处产生一极大峰值,在实验误差允
    许范围内,与理论计算所得的图像一致。

    谐振频率约为2.25KHz,计算出的Q值为11.93,所做的相频特性曲线与幅频特性曲线都与实验原理中的理论图像符合情况较好,说明实验完成情况较好。
        

    \subsection{测量RLC并联电路的相频特性和幅频特性曲线}
    \subsubsection{找到谐振频率}
    调节信号发生器频率,得到谐振频率为$f=2.25kHz$
    \subsubsection{测量相频特性曲线和幅频特性曲线}
    \begin{table}[H]
        \centering
        \begin{tabular}{|l|l|l|l|l|l|l|}
        \hline
            f/kHz & U/V & $t/\mu s$ & $\varphi^{\circ}$ & U/V & $U_r/mV$ & $I_max/mA$ \\ \hline
            2.05 & 2 & 86 & 63.47  & 1.58 & 1080  & 0.216  \\ \hline
            2.15 & 2 & 84 & 65.02  & 1.82 & 840  & 0.168  \\ \hline
            2.2 & 2 & 78 & 61.78  & 1.86 & 380 & 0.076  \\ \hline
            2.231 & 2 & 50 & 40.16  & 1.9 & 260  & 0.052  \\ \hline
            2.24 & 2 & 34 & 27.42  & 1.9 & 220  & 0.044  \\ \hline
            2.247 & 2 & 10 & 8.09  & 1.88 & 160  & 0.032  \\ \hline
            2.25 & 2 & 0 & 0 & 1.86 & 154  & 0.031  \\ \hline
            2.253 & 2 & -12 & -9.73  & 1.86 & 140  & 0.028  \\ \hline
            2.256 & 2 & -20 & -16.24  & 1.86 & 148  & 0.030  \\ \hline
            2.265 & 2 & -50 & -40.77  & 1.86 & 180  & 0.036  \\ \hline
            2.275 & 2 & -80 & -65.52  & 1.86 & 216  & 0.043  \\ \hline
            2.32 & 2 & -86 & -71.83  & 1.84 & 432  & 0.086  \\ \hline
            2.4 & 2 & -100 & -86.4  & 1.74 & 784 & 0.157 \\ \hline
            2.6 & 2 & -110 & -103  & 1.74 & 784 & 0.157 \\ \hline
        \end{tabular}
    \end{table}

    绘制图像:
    \begin{figure}[H]
        \centering
        \includegraphics[scale=0.75]{图片3.png}
    \end{figure}
    \begin{figure}[H]
        \centering
        \includegraphics[scale=0.75]{图片4.png}
    \end{figure}

    \subsubsection{讨论}
    我们绘制的相频特性曲线和课本上的理论曲线大体相符,但仍有有一定差别。
    可以观察得到随着频率的变化,相位差由正到负,在 90°附近现出逐渐平行于 x 轴的趋势。

    幅频特性曲线中,电压先上升再下降,电流则刚好相反,且二者的极值点都对应一个相同的
    值,(该值虽然不严格等于谐振频率,但十分接近谐振频率)并
    且除谐振频率附近的时候有波动外,整体光滑平稳,在实验误差允许范围内,符合理论预期。

    误差产生可能的原因如下:

    我们选取的频率范围基本在谐振频率附近,并未测量频率较小时的部分。

    并联电路与串联电路最大的区别在于,不能直接从示波器的 CH1 或 CH2 的曲线中读取有关电压
    的变化曲线,必须将这两个通道相减,读数误差较大,所以导致幅频曲线误差较大。

    在并联电路读取相位差时,不能依靠系统自动计算,只能使用光标去测定.用肉眼观察操作时容易造成较大误差
    
    与串联时不同,相位差还要通过公式去计算:信号发生器的频率 f 本身就有一定误差,我们测得
    的时间间隔$\Delta t$ 也有误差,两者的乘法运算则会导致更大的误差

    
    \subsection{观测RLC串联电路的暂态过程}
    \subsubsection{$R=0\Omega$时测量$u_C$波形}
    \begin{figure}[H]
        \centering
        \includegraphics[scale=0.9]{20.jpg}
    \end{figure}
    图像的确呈现出逐渐衰减的震荡形式,在与理论预期相符

    \subsubsection{调节R测得临界电阻,并与理论值比较}

    \begin{figure}[H]
        \centering
        \includegraphics[scale=0.3]{24.png}
    \end{figure}

    理论值$R_C=\sqrt{\frac{\Delta L}{C}}=1414\Omega$,测量值为$1190\Omega$。与理论值有
    一定的误差,可能的原因是接近临界电阻时,图像随电阻的变化越来越不明显,导致难以准确测得临界阻值。

    \subsubsection{记录$R=2k\Omega,20k\Omega$的$u_C$波形,函数发生器频率分别为$250Hz$和$20Hz$}
    \begin{figure}[H]
        \centering
        \caption*{$R=20k\Omega,f=20Hz$}
        \includegraphics[scale=0.9]{21.jpg}
    \end{figure}


    \subsubsection{讨论}
    将实验测得的三个图像分别与理论计算的三个图像对比,在实验误差允许范围内符合的较好。\par
    可能产生误差的原因是接近临界电阻时,图像随电阻的变化越来越不明显,导致难以准确测得临界阻值。即示波器的
    分辨率有限导致误差。
    
\section{实验总结与反思}
    \subsection{有关实验操作细节与技巧}
    (1)本次的三个实验,务必要按照实验原理的电路图连接。由于示波器与信号发生器都接地,导致
    要测得实验需要的数据,使用特定连接方式会大大便利实验的测量。\par
    (2)由于函数发生器内阻的原因,测量时要时刻注意调节输入信号的幅度,维持外电路总电压峰峰
    值为 2V 左右,使得我们的实验符合理论要求。\par
    (3)在用示波器测量时,需要使用示波器统计功能。这里的技巧是:每次测量最好重新开启统计的
    按钮,使示数尽快平稳,极大程度地缩短了实验时间。\par

    \subsection{有关数据记录与处理的细节}
    (1)注意不要读取峰峰值,而是读取幅度值,主要是因为是峰峰值是未经过处理的信
    号,其中噪音成分含量过多。因此,采用幅度的实验数据会更加可靠。\par
    (2)测量并联谐振时,只能用光标来测量时间差进而计算相位差,但曲线与 x 轴交点有宽度,并且
    曲线有时会变得很平直,这时我们可以放大图像,并统一标准,把光标对准宽度中心点的位置,这样能
    够最大限度的消除由于主观估计带来的误差。



    \subsection{反思}
    (1)实验前务必要预习,尤其是透彻理解实验理论与原理\par
    (2)灵活使用计算机软件是处理实验数据必不可少的技能\par
    (3)认真对待误差分析\par
    (4)多请教老师对于快速掌握实验技巧很有帮助\par

    \section*{附:原始实验数据}
    \begin{figure}[H]
        \centering
        \includegraphics[scale=0.15]{24.jpg}
        \includegraphics[scale=0.15]{25.jpg}
    \end{figure}

\end{document}