\documentclass[11pt,a4paper]{article}
\usepackage[T1]{fontenc}
\usepackage{amsmath}
\usepackage{amssymb}
\usepackage{graphicx}
\usepackage[UTF8,heading=true]{ctex}
\usepackage{geometry}
\usepackage{diagbox}
\usepackage[]{float}
\usepackage{xeCJK}
\usepackage{indentfirst}
\usepackage{multirow}
\usepackage[section]{placeins}
\usepackage{caption}
\usepackage{cite}
\usepackage{graphics}
\usepackage{subfig}

\graphicspath{{./figure/}}

\setCJKfamilyfont{zhsong}[AutoFakeBold = {5.6}]{STSong}
\newcommand*{\song}{\CJKfamily{zhsong}}

\geometry{a4paper,left=2cm,right=2cm,top=0.75cm,bottom=2.54cm}

\newcommand{\experiName}{光学基础实验}%实验名称
\newcommand{\name}{张钰堃}
\newcommand{\studentNum}{2022K8009926020}
\newcommand{\dateYear}{2023}
\newcommand{\dateMonth}{9}%月
\newcommand{\dateDay}{12}%日
\newcommand{\room}{教学楼705}%地点
\newcommand{\others}{$\square$}

\ctexset{
    section={
        format+=\raggedright\song\large
    },
    subsection={
        name={\quad,.}
    },
    subsubsection={
        name={\qquad,.}
    }
}

\begin{document}
\noindent

\begin{center}

    \textbf{\song \zihao{-2} \ziju{0.5}《基础物理实验》实验报告}
    
\end{center}


\begin{center}
    \kaishu \zihao{5}
    \noindent \emph{实验名称}\underline{\makebox[14em][c]{\experiName}}
    \emph{姓名}\underline{\makebox[6em][c]{\name}} 
    \emph{学号}\underline{\makebox[14em][c]{\studentNum}}
    \emph{实验日期} \underline{\makebox[3em][c]{\dateYear}} \emph{年}
    \underline{\makebox[2em][c]{\dateMonth}}\emph{月}
    \underline{\makebox[2em][c]{\dateDay}}\emph{日}
    \emph{实验地点}\underline{{\makebox[6em][c]\room}}
    \emph{调课/补课} \underline{\makebox[3em][c]{否}}
    \emph{成绩评定} \underline{\hspace{8em}}
    {\noindent}
    \rule[5pt]{17.7cm}{0.2em}

\end{center}

\section{实验目的及要求}
1、了解与学习激光产生的基本原理以及传播和接收等基本特性。

2、观测激光传输、扩束等实验现象。

3、通过搭建马赫—曾德干涉仪掌握激光光路的基本调节方法。

4、通过检偏器学习激光偏振态的检验。

5、观察夫琅和费衍射和光栅衍射现象。


\section{实验仪器}
He-Ne激光器、反射镜镜架、加强铝反射镜、棱镜架、分光镜、光栅、透镜架、透镜、套筒、偏振片架、偏振片、支杆,内六角螺丝、内六角扳手、光学平板

\section{实验内容}
\subsection{M-Z干涉仪}
\begin{figure}[H]
    \centering
    \includegraphics[scale=0.8]{1.png}
    \caption{光路图}
\end{figure}
\begin{figure}[H]
    \centering
    \includegraphics[scale=0.15]{M-Z干涉仪1.png}
    \includegraphics[scale=0.15]{M-Z干涉仪器2.png}
    \caption{光路实现}
\end{figure}
\begin{figure}[H]
    \centering
    \includegraphics[scale=0.15]{干涉条纹1.png}
    \includegraphics[scale=0.15]{干涉条纹2.png}
    \caption{干涉效果}
\end{figure}
在老师的帮助下,经过反复地尝试与调整,我们最后在面板上观察到如图所示的干涉条纹,条纹相对清晰。
\subsection{光的衍射}
\begin{figure}[H]
    \centering
    \includegraphics[scale=0.2]{矩孔衍射.png}
    \caption{矩孔衍射}
\end{figure}
\begin{figure}[H]
    \centering
    \includegraphics[scale=0.2]{双缝衍射.png}
    \caption{双缝衍射}
\end{figure}
\begin{figure}[H]
    \centering
    \includegraphics[scale=0.2]{圆孔衍射.png}
    \caption{圆孔衍射}
\end{figure}
通过不断尝试,获得了十分清晰的衍射图样
\subsection{用偏振片验证马吕斯公式}
\subsubsection{实验数据表格}
\begin{figure}[H]
    \centering
    \includegraphics[scale=1]{2.png}    
\end{figure}
\subsubsection{数据分析}
\begin{figure}[H]
    \centering
    \includegraphics[scale=1]{3.png}
\end{figure}
\begin{figure}[H]
    \centering
    \includegraphics[scale=0.8]{10.jpg}
    \caption{理想条件下的实验结果}
\end{figure}
(该图片摘取自$http://www.fanwen118.com/info_24/fw_3705325.html$)

本次实验只测得了0-90度范围内的数据,与预测结果图像在图形上较为相似,但极值点偏差较大。误差原因可能有:偏振片不够平整;偏振片倾斜角度存在误差;角度测量存在误差等

\end{document}