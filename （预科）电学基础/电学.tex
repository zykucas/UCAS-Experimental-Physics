\documentclass[11pt,a4paper]{article}
\usepackage[T1]{fontenc}
\usepackage{amsmath}
\usepackage{amssymb}
\usepackage{graphicx}
\usepackage[UTF8,heading=true]{ctex}
\usepackage{geometry}
\usepackage{diagbox}
\usepackage[]{float}
\usepackage{xeCJK}
\usepackage{indentfirst}
\usepackage{multirow}
\usepackage[section]{placeins}
\usepackage{caption}
\usepackage{cite}
\usepackage{graphics}
\usepackage{subfig}

\graphicspath{{./figure/}}

\setCJKfamilyfont{zhsong}[AutoFakeBold = {5.6}]{STSong}
\newcommand*{\song}{\CJKfamily{zhsong}}

\geometry{a4paper,left=2cm,right=2cm,top=0.75cm,bottom=2.54cm}

\newcommand{\experiName}{简单电学实验}%实验名称
\newcommand{\name}{张钰堃}
\newcommand{\studentNum}{2022K8009926020}
\newcommand{\dateYear}{2023}
\newcommand{\dateMonth}{10}%月
\newcommand{\dateDay}{10}%日
\newcommand{\room}{教学楼702}%地点
\newcommand{\others}{$\square$}

\ctexset{
    section={
        format+=\raggedright\song\large
    },
    subsection={
        name={\quad,.}
    },
    subsubsection={
        name={\qquad,.}
    }
}

\begin{document}
\noindent

\begin{center}

    \textbf{\song \zihao{-2} \ziju{0.5}《基础物理实验》实验报告}
    
\end{center}


\begin{center}
    \kaishu \zihao{5}
    \noindent \emph{实验名称}\underline{\makebox[14em][c]{\experiName}}
    \emph{姓名}\underline{\makebox[6em][c]{\name}} 
    \emph{学号}\underline{\makebox[14em][c]{\studentNum}}
    \emph{实验日期} \underline{\makebox[3em][c]{\dateYear}} \emph{年}
    \underline{\makebox[2em][c]{\dateMonth}}\emph{月}
    \underline{\makebox[2em][c]{\dateDay}}\emph{日}
    \emph{实验地点}\underline{{\makebox[4em][c]\room}}
    \emph{调课/补课} \underline{\makebox[3em][c]{否}}
    \emph{成绩评定} \underline{\hspace{8em}}
    {\noindent}
    \rule[5pt]{17.7cm}{0.2em}

\end{center}

\section{实验目的及要求}
1.测量非线性元件的伏安曲线

2.整流电路和电容特性

\section{实验仪器}
发光LED/二极管1N4007/稳压二极管、小灯泡、数字电源、模拟横流恒压电源、模拟电源、万用表、转接头、面包板
\section{实验内容}
\subsection{测量发光二极管的伏安曲线}
1、使用万用表测量电阻阻值和二极管元件的正向导通压降。并估算、选用电源电压、电阻。

2、设计、搭建电路,选择合适的电源电压和万用表档位、量程。

3、测量、绘制伏安特性曲线
\subsection{整流滤波电路}
1、用万用表测量电容值。

2、搭建二极管整流桥电路。使用一个电阻作为负载。观察、比较滤波前后的波形和幅值变化。

3、在电路中加入不同大小的电容进行滤波,比较大电容、有电容、无电容的波形变化。

4、用信号发生器产生方波,用不同的电阻与电容串联,观察充放电过程、计算充电常数。


\section{实验电路设计与数据分析}
\subsection{测量发光二极管的伏安曲线}
\subsubsection{电路设计}
\begin{figure}[H]
    \centering
    \includegraphics[scale=0.8]{图片1.png}
    \caption{电路图}
\end{figure}
\begin{figure}[H]
    \centering
    \includegraphics[scale=0.5]{1.png}
    \caption{电路实现}
\end{figure}
数据处理:
\begin{table}[H]
    \centering
    \begin{tabular}{|l|l|l|l|l|l|l|1|1|1|1|1|1|}
    \hline
        电压(V) &1.72 & 1.74&1.76&1.77&1.78&1.79&1.80&1.83&1.85&1.90&1.95&2.00 \\ \hline
        电流(mA) & 0.11 & 0.21&0.33&0.42&0.53&0.70&0.95&1.94&2.40&5.75&10.52&16.87 \\ \hline
    \end{tabular}
\end{table}

\begin{figure}[H]
    \centering
    \includegraphics[scale=0.8]{3.png}
    \caption{原始数据记录}
\end{figure}
数据中有两组疑似为记录错误,在统计时去除
\begin{figure}[H]
    \centering
    \includegraphics[scale=0.8]{pic3.png}
    \caption{数据分析}
\end{figure}
上图说明发光二极管的伏安特性确实存在指数关系,与公式$I=I_s(e^{\frac{U}{U_r}}-1)$吻合
\subsubsection{整流滤波电路}
\begin{figure}[H]
    \centering
    \includegraphics[scale=0.5]{pic6.png}
    \includegraphics[scale=0.4]{pic5.png}
    \caption{半波整流与全波整流电路}
\end{figure}
\begin{figure}[H]
    \centering
    \includegraphics[scale=0.4]{4.png}
    \caption{桥式整流电路}
\end{figure}
\begin{figure}[H]
    \centering
    \includegraphics[scale=0.5]{5.png}
    \caption{整流结果}
\end{figure}
实际实验时,利用桥式整流电路并不能得到完美的波形,这是因为二极管存在导通电压,在输入电压低于或约等于导通电压时,二极管
不会导通(电阻趋于无穷大),因此电阻不会有分压,导致波形低于导通电压的部分被截断
\end{document}
