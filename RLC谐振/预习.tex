\documentclass[11pt,a4paper]{article}
\usepackage[T1]{fontenc}
\usepackage{amsmath}
\usepackage{amssymb}
\usepackage{graphicx}
\usepackage[UTF8,heading=true]{ctex}
\usepackage{geometry}
\usepackage{diagbox}
\usepackage[]{float}
\usepackage{xeCJK}
\usepackage{indentfirst}
\usepackage{multirow}
\usepackage[section]{placeins}
\usepackage{caption}
\usepackage{cite}
\usepackage{graphics}
\usepackage{subfig}

\graphicspath{{./figure/}}

\setCJKfamilyfont{zhsong}[AutoFakeBold = {5.6}]{STSong}
\newcommand*{\song}{\CJKfamily{zhsong}}

\geometry{a4paper,left=2cm,right=2cm,top=0.75cm,bottom=2.54cm}

\newcommand{\experiName}{RLC电路的谐振与暂态过程}%实验名称
\newcommand{\supervisor}{黄金浩}%指导教师
\newcommand{\name}{张钰堃}
\newcommand{\studentNum}{2022k8009926020}
\newcommand{\class}{2}%班级
\newcommand{\group}{08}%组
\newcommand{\seat}{11}%座位号
\newcommand{\dateYear}{2023}
\newcommand{\dateMonth}{10}%月
\newcommand{\dateDay}{24}%日
\newcommand{\room}{教学楼709}%地点
\newcommand{\others}{$\square$}

\ctexset{
    section={
        format+=\raggedright\song\large
    },
    subsection={
        name={\quad,.}
    },
    subsubsection={
        name={\qquad,.}
    }
}

\begin{document}
\noindent

\begin{center}

    \textbf{\song \zihao{-2} \ziju{0.5}《基础物理实验》预习报告}
    
\end{center}


\begin{center}
    \kaishu \zihao{5}
    \noindent \emph{实验名称}\underline{\makebox[28em][c]{\experiName}}
    \emph{指导教师}\underline{\makebox[9em][c]{\supervisor}}\\
    \emph{姓名}\underline{\makebox[6em][c]{\name}} 
    \emph{学号}\underline{\makebox[14em][c]{\studentNum}}
    \emph{分班分组及座号} \underline{\makebox[5em][c]{\class \ -\ \group \ -\ \seat }\emph{号}} (\emph{例}:\,1- 04- 5\emph{号})\\
    \emph{实验日期} \underline{\makebox[3em][c]{\dateYear}} \emph{年}
    \underline{\makebox[2em][c]{\dateMonth}}\emph{月}
    \underline{\makebox[2em][c]{\dateDay}}\emph{日}
    \emph{实验地点}\underline{{\makebox[4em][c]\room}}
    \emph{调课/补课} \underline{\makebox[3em][c]{否}}
    \emph{成绩评定} \underline{\hspace{8em}}
    {\noindent}
    \rule[5pt]{17.7cm}{0.2em}
\end{center}

\section{实验目的}
1. 研究 RLC 电路的谐振现象。

2. 了解 RLC 电路的相频特性和幅频特性。

3. 用数字存储示波器观察 RLC 串联电路的暂态过程,理解阻尼振动规律。
\section{实验仪器与用具}
标准电感,标准电容,100$\Omega $,标准电阻,电阻箱,电感箱,电容箱,函数发生器,示波器,
数字多用表,导线等。
\section{实验原理}
\subsection{串联谐振}
RLC 串联电路如图 1 所示。其总阻抗$\left\lvert Z\right\rvert $、电压u与电流i之间的相位差$\varphi $、电流i分别为
\begin{equation}
    \left\lvert Z\right\rvert=\sqrt{R^2+(\omega L-\frac{1}{\omega C} )^2}  
\end{equation}
\begin{equation}
    \varphi =\arctan \frac{\omega L-\frac{1}{\omega C} }{R} 
\end{equation}
\begin{equation}
   i= \frac{u}{\sqrt{R^2+(\omega L-\frac{1}{\omega C} )} } 
\end{equation}
式中$\omega =2\pi f$为角频率,$\left\lvert Z\right\rvert $、$\varphi$ 、i都是f的函数
\begin{figure}[H]
    \centering
    \includegraphics[scale=0.7]{1.png}
    \caption{RLC串联电路}
\end{figure}
图 2(a)、(b)、(c)分别为 RLC 串联电路的阻抗、相位差、电流随频率的变化曲线。其中图 2(b)$\varphi -f$曲线称为相频特性曲线;图 2(c)
$i-f$曲线称为幅频特性曲线,它表示在总电压u保持不变的条件下i随f的变化曲线。相频特性曲线和幅频特性曲线有时统称为频率响应特
性曲线。

由曲线图可以看出,存在一个特殊的频率$f_0$,特点为:
(1)当$f<f_0$时,$\varphi <0$,电流的相位超前于电压,整个电路呈电容性,且随f降低,$\varphi$ 趋近于-$\frac{\pi }{2}$;而当$f>f_0$时,$\varphi >0$,电流的相位落后于电压,整个电路呈电感性,且随
f升高,$\varphi$ 趋近于$\frac{\pi }{2}$
(2)随f偏离$f_0$越远,阻抗越大,而电流越小。
(3)当$\omega L-\frac{1}{\omega C}$,即
\begin{equation}
    \omega _0=\frac{1}{\sqrt{LC} } 或 f_0=\frac{1}{2\pi \sqrt{LC} }
\end{equation}
$\varphi  = 0$,电压与电流同相位,整个电路呈纯电阻性,总阻抗达到极小值$Z_0=R$,而总电流达到极大值$i_m=\frac{u}{R}$。这种特殊的状态称为串联谐振,此时角频率
$\omega _0$称为谐振角频率。在$f_0$处,$i − f$曲线有明显尖锐的峰显示其谐振状态,因此,有时称它为谐振曲线。谐振时,有
\begin{equation}
    u_L=i_m\left\lvert Z_L\right\rvert =\frac{{\omega _0}Lu}{R},\frac{u_L}{u}=\frac{{\omega _0}L}{R}=\frac{1}{R}*\sqrt{\frac{L}{C}}
\end{equation}
而
\begin{equation}
    u_C=i_m\left\lvert Z_C\right\rvert =\frac{u}{RC{\omega _0}},\frac{u_C}{u}=\frac{1}{RC{\omega _0}}=\frac{1}{R}*\sqrt{\frac{L}{C}}
\end{equation}
令
\begin{equation}
    Q=\frac{u_C}{u}=\frac{u_L}{u}=\frac{{\omega _0}L}{R}=\frac{1}{RC{\omega _0}}
\end{equation}
Q称为谐振电路的品质因数,简称Q值。它是由电路的固有特性决定的,是标志和衡量谐振电路性能优劣的重要的参量。
\subsection{并联谐振}
如图 3 所示电路,其总阻抗$\left\lvert Z_P\right\rvert$、电压u与电流i之间的相位差$\varphi $、电压u(或电流i)分别为
\begin{equation}
    \left\lvert Z_P\right\rvert=\sqrt{\frac{R^2+(\omega L)^2}{(1-{\omega }^2LC)^2}+(\omega CR)^2} 
\end{equation}
\begin{equation}
    \arctan {\frac{\omega L-\omega C[R^2+(\omega L)^2]}{R}}    
\end{equation}
\begin{equation}
    u-i\left\lvert Z_P\right\rvert =\frac{u_R}{u}\left\lvert Z_P\right\rvert 
\end{equation}
显然,它们都是频率的函数。当$\varphi $=0 时,电流和电压同相位,整个电路呈纯电阻性,即发生谐振。由式(8)求得并联谐振的角频率$\omega _p$
为
\begin{equation}
    \omega _p=2\pi f_p=\sqrt{\frac{1}{LC}-(\frac{R}{L})^2} =\omega _0\sqrt{1-\frac{1}{Q^2}} 
\end{equation}
式中$\omega _0=2\pi f_0=\frac{1}{LC},Q=\frac{\omega _0L}{R}=\frac{1}{R}\sqrt{\frac{L}{C}}$,可见,并联谐振频率$f_p$与$f_0$稍有不同,
当Q >> 1时,$\omega _p\approx \omega _0$, $f_P\approx f_0$。

图 4(a)、(b)、(c)分别为 RLC 并联电路的阻抗、相位差、电流或电压随频率的变化曲线。
由(b)图$\varphi  − f$曲线可见,在谐振频率$f=f_P$两侧,当$f<f_P$,$\varphi >0$,电流的相位落后于电压,整个电路呈电感性;当
$f>f_P$,$\varphi <0$,电流的相位超前于电压,整个电路呈电容性。

显然,在谐振频率两边区域,并联电路的电抗特性与串联电路时截然相反。由(a)图$\left\lvert Z_p\right\rvert-f$线和(c)图i − f
曲线可见,在$f={f_p}^'$处(注意:${f_p}^'$与$f_p$稍有不同)总阻抗达到极大值,总电流达到极小值,而在${f_p}^'$两侧,随f偏离
${f_p}^'$越远,阻抗越小,电流越大。不言而喻,这种特性,与串联电路时完全相反。(c)图u − f曲线为在总电流保持不变的条件下,电感(或电容)两端电压
u随频率的变化曲线。
\begin{figure}[H]
    \centering
    \includegraphics[scale=0.7]{4.png}
\end{figure}
与串联谐振类似,可用品质因数Q,即
\begin{equation}
    Q_1=\frac{\omega _0L}{R}=\frac{1}{R\omega _0C};Q_2=\frac{i_c}{i}\approx \frac{i_L}{i};Q_3=\frac{f_0}{\Delta f}
\end{equation}
标志并联谐振电路的性能优劣,其意义也类同。不过,此时$i_L\approx i_C = iQ$,谐振支路中的电流为总电流的Q倍。因此,有时称并联谐振为电流谐振。
\subsection{ RLC 电路的暂态过程}
电路如图 5。先观察放电过程,即开关 S 先合向“1”使电容充电至 E,然后把 S 倒向“2”,电容就在闭合的
RLC 电路中放电。电路方程为
\begin{figure}[H]
    \centering
    \includegraphics[scale=0.7]{5.png}
\end{figure}
\begin{figure}[H]
    \centering
    \includegraphics[scale=0.7]{19.png}
\end{figure}
又将$ i = C\frac{d{u_C}}{dt}$代入得
\begin{figure}[H]
    \centering
    \includegraphics[scale=0.7]{8.png}
\end{figure}
根据初始条件t = 0,$u_c = E$,$\frac{d{u_c}}{dt}=0$解方程。方程的解分为 3 种情况:

(1)$R^2<\frac{4L}{C}$属于阻尼较小的情况。引入阻尼系数$\zeta  =\frac{R}{2}\sqrt{\frac{C}{L}} $𝐿
后,对应于$\zeta $< 1。此时方程的解为
\begin{figure}[H]
    \centering
    \includegraphics[scale=0.8]{7.png}
\end{figure}
其中时间常量为
\begin{figure}[H]
    \centering
    \includegraphics[scale=0.8]{9.png}
\end{figure}
衰减振动的角频率为
\begin{figure}[H]
    \centering
    \includegraphics[scale=0.8]{10.png}
\end{figure}
$U_C$随时间变化的规律如图曲线 I 所示,即阻尼振动状态。此时振动的振幅呈指数衰减。$\tau $的
大小决定了振幅衰减的快慢,$\tau $越小,振幅衰减越迅速。
\begin{figure}[H]
    \centering
    \includegraphics[scale=0.6]{11.png}
\end{figure}
如果$R^2<<\frac{4L}{C}$,通常是 R 很小的情况,振幅的衰减很缓慢
\begin{figure}[H]
    \centering
    \includegraphics[scale=0.8]{12.png}
\end{figure}
此时近似为 LC 电路地自由振动,$\omega _0$为 R=0 时 LC 回路的固有频率。衰减振动的周期
\begin{figure}[H]
    \centering
    \includegraphics[scale=0.8]{13.png}
\end{figure}
(2)$R^2>\frac{4L}{C}$,即阻尼系数$\zeta $> 1,1。对应于过阻尼状态,其解为
\begin{figure}[H]
    \centering
    \includegraphics[scale=0.8]{14.png}
    \includegraphics[scale=0.8]{15.png}
\end{figure}
式所表示的$u_c$ − 𝑡的关系曲线见图 6 中的曲线 II,它是以缓慢的方式逐渐回零。可以证明,
若 L 和 C 固定,随电阻 R 的增长,$u_c$衰减到零的过程更加缓慢。
(3)$R^2=\frac{4L}{C}$,即阻尼系数$\zeta $= 1。对应于临界阻尼状态,其解为
\begin{figure}[H]
    \centering
    \includegraphics[scale=0.8]{16.png}
\end{figure}
其中$\tau $ = 2𝐿/𝑅。它是从过阻尼到阻尼振动过渡的分界点,$u_c$ − 𝑡的关系见图 6 中的曲线 III。
对于充电过程,即开关 S 先在位置“2”,待电容放电完毕,再把 S 倒向“1”,电源 E 将对电
容充电,于是电路方程变为
\begin{figure}[H]
    \centering
    \includegraphics[scale=0.8]{17.png}
\end{figure}
初始条件为t = 0 时,$u_c$ = 0,$\frac{d{u_c}}{dt} = 0$。方程解为
\begin{figure}[H]
    \centering
    \includegraphics[scale=0.8]{18.png}
\end{figure}
\end{document}