\documentclass[11pt,a4paper]{article}
\usepackage[T1]{fontenc}
\usepackage{amsmath}
\usepackage{amssymb}
\usepackage{graphicx}
\usepackage[UTF8,heading=true]{ctex}
\usepackage{geometry}
\usepackage{diagbox}
\usepackage[]{float}
\usepackage{xeCJK}
\usepackage{indentfirst}
\usepackage{multirow}
\usepackage[section]{placeins}
\usepackage{caption}
\usepackage{cite}
\usepackage{graphics}
\usepackage{subfig}

\graphicspath{{./figure/}}

\setCJKfamilyfont{zhsong}[AutoFakeBold = {5.6}]{STSong}
\newcommand*{\song}{\CJKfamily{zhsong}}

\geometry{a4paper,left=2cm,right=2cm,top=0.75cm,bottom=2.54cm}

\newcommand{\experiName}{示波器等的使用}%实验名称
\newcommand{\name}{张钰堃}
\newcommand{\studentNum}{2022K8009926020}
\newcommand{\dateYear}{2023}
\newcommand{\dateMonth}{9}%月
\newcommand{\dateDay}{19}%日
\newcommand{\room}{教学楼702}%地点
\newcommand{\others}{$\square$}

\ctexset{
    section={
        format+=\raggedright\song\large
    },
    subsection={
        name={\quad,.}
    },
    subsubsection={
        name={\qquad,.}
    }
}

\begin{document}
\noindent

\begin{center}

    \textbf{\song \zihao{-2} \ziju{0.5}《基础物理实验》实验报告}
    
\end{center}


\begin{center}
    \kaishu \zihao{5}
    \noindent \emph{实验名称}\underline{\makebox[17em][c]{\experiName}}
    \emph{姓名}\underline{\makebox[6em][c]{\name}} 
    \emph{学号}\underline{\makebox[14em][c]{\studentNum}}
    \emph{实验日期} \underline{\makebox[3em][c]{\dateYear}} \emph{年}
    \underline{\makebox[2em][c]{\dateMonth}}\emph{月}
    \underline{\makebox[2em][c]{\dateDay}}\emph{日}
    \emph{实验地点}\underline{{\makebox[4em][c]\room}}
    \emph{调课/补课} \underline{\makebox[3em][c]{否}}
    \emph{成绩评定} \underline{\hspace{8em}}
    {\noindent}
    \rule[5pt]{17.7cm}{0.2em}

\end{center}

\section{实验目的及要求}
1. 了解模拟示波器显示波形的主要原理,和数字示波器的主要差异。

2. 学会示波器和信号发生器的一般使用方法。

3. 学会用示波器测量电压波形幅度、频率和相位差[间隔]等参数。

4、 学会正确的使用万用表和直流电源。

5、 学会用万用表测电阻、电压、电容和判断二极管的极性。

6、 理解直流电源的恒压模式和恒流模式的切换机制和意义。

7、 学会用直流电源测二极管的伏安特性,用示波器观察直流电源的纹波。

\section{实验仪器}
数字存储示波器( DS1102E, MSO1104Z, MSO2302A, SDS1104X-E,Tektronix
MSO2022B)、信号发生器(DG1022U,DG4162)、信号电路板(DS1000D-TK)、稳压电源(DP832、
DP711、DH1715)、线材和工具(示波器探头、BNC 信号线、BNC 转鳄鱼夹信号线、方口 USB
线、电容无感螺丝批)、DHJX-1 简谐振动合成仪、DSG815 射频信号源、电子元件(LED、光电池、线圈喇叭、压电陶瓷
片等)、万用表(Fluke 17B+,优利德 UT39A)、电源(HZDH DH1715A, rigol DP832,
rigol DP711)、功率电阻。

\section{实验步骤与数据记录}
\subsection{调出稳定波形}
\subsubsection{重要步骤}
1.设置信号发生器产生一个频率为 1 KHz,峰峰值电压 VPP=4 V 的正弦波信号,用示波
器进行观察。(使用信号发生器 CH1 通道产生信号输出到示波器的 CH1 通道。点击示波器上的 AUTO 键,
使显示出清晰稳定的波形。调节垂直控制(VERTICAL)和水平控制(HORIZONTAL)区域中
的 Scale 和 Position 旋钮,观察显示的图像有何变化。调节触发区域的触发电平(LEVEL)旋钮,
观察图像有何变化。)

2.由信号发生器产生一个频率为 1 KHz,峰峰值电压 VPP=4 V 的方波(然后三角波)信
号。调节产生波形的参数,观察波形变化。

3.由信号发生器产生两个频率为 1 KHz,峰峰值电压 VPP=4 V 的正弦波信号,经示波器
显示出来,观察示波器图像的变化
\subsubsection{实验数据记录}
\begin{figure}[H]
    \centering
    \includegraphics[scale=0.2]{picture1.jpg}
    \includegraphics[scale=0.2]{picture2.jpg}
    \caption{利用信号发生器产生 1kHz, Vpp = 4V 的正弦波信号,观察到的图像如图}
\end{figure}
\begin{figure}[H]
    \centering
    \includegraphics[scale=0.2]{picture3.jpg}
    \caption{利用信号发生器产生 1kHz, Vpp = 4V 的方波信号,观察到的图像如图}
\end{figure}
\begin{figure}[H]
    \centering
    \includegraphics[scale=0.2]{picture4.jpg}
    \includegraphics[scale=0.2]{picture5.jpg}
    \includegraphics[scale=0.2]{picture6.jpg}
    \caption{利用信号发生器产生两个正弦波信号,观察到的图像如图}
\end{figure}
\subsection{电压、时间间隔和频率的测量}
\subsubsection{重要步骤}
1.利用示波器屏幕上的标尺测量信号的周期、频率、电压峰-峰值和有效值。

2.利用测量功能(示波器前面板上的“Measure”键)测量同一信号的周期、频率、电压峰
-峰值和有效值,并与(1)的测量结果进行比较。
\subsubsection{实验数据记录}
\begin{figure}[H]
    \centering
    \includegraphics[scale=0.4]{picture7.png}
    \includegraphics[scale=0.2]{picture8.jpg}
    \caption{电压、时间间隔和频率的测量}
\end{figure}
\subsection{波形的运算}
    \subsubsection{重要步骤}
    将示波器 CH1 输入 VPP=5 V,频率为 1 KHz 的正弦波信号,CH2 通道输入同等幅值、频率
    为 3 KHz 的正弦波。利用控制面板上的“MATH”按键,将两路波形相加,观察相加后的波形。将
    CH2 的信号频率改为 5 KHz、10 KHz 后,观察相加后的波形。
    \subsubsection{实验数据记录}
    \begin{figure}[H]
        \centering
        \includegraphics[scale=0.2]{picture9.jpg}
        \caption{波形的运算}
    \end{figure}
\subsection{观测李萨如图形}
\subsubsection{重要步骤}
切换至示波器的 X-Y 模式(“HORIZONTAL”区域,“Menu”按键,“时基”),示波器 CH1 和
CH2 通道中分别接入 VPP=5 V,频率分别为(1 KHz, 1 KHz)、(1 KHz, 2 KHz)、(1 KHz,
3 KHz)、(2 KHz, 3 KHz)的信号,观察李萨如图形。改变 CH2 的相位,观察图形的变化。画
出(CH1,CH2)频率分别为(1 KHz, 4 KHz)、(2 KHz, 5 KHz)的李萨如图形。
\subsubsection{实验数据记录}
\begin{figure}[H]
    \centering
    \includegraphics[scale=0.2]{pic1.jpg}
    \includegraphics[scale=0.2]{pic2.jpg}
    \caption{李萨如图形}
\end{figure}
\begin{figure}[H]
    \centering
    \includegraphics[scale=0.2]{pic3.jpg}
    \includegraphics[scale=0.2]{pic4.jpg}
    \caption{李萨如图形}
\end{figure}
\begin{figure}[H]
        \centering
        \includegraphics[scale=0.2]{pic5.jpg}
        \includegraphics[scale=0.2]{pic6.jpg}
        \caption{李萨如图形}
\end{figure}
\begin{figure}[H]
        \centering
        \includegraphics[scale=0.2]{pic7.jpg}
        \includegraphics[scale=0.2]{pic8.jpg}
        \caption{李萨如图形}
\end{figure}
\subsection{万用表的使用}
\subsubsection{重要步骤}
1.用万用表测两手之间的身体电阻。

2.切换万用表到通断模式,将两个表笔短接,听短路时的嘀嘀声。

3.切换万用表到二极管模式,测二极管的截止电压和电阻,判断二极管的极性。

4.切换万用表到电容模式,测电容器的电容。

5.切换万用表到交流档,测插座的交流电压,

6.将电源采用恒压模式输出 5V,12V,24V,用万用表测试输出的电压值。
\subsubsection{实验数据记录}
\begin{figure}[H]
    \centering
    \includegraphics[scale=0.2]{pic12.jpg}
    \includegraphics[scale=0.2]{pic13.jpg}
    \caption{万用表的使用}
    
\end{figure}
\begin{figure}[H]
\includegraphics[scale=0.2]{pic14.jpg}
\includegraphics[scale=0.2]{pic15.jpg}
\includegraphics[scale=0.2]{pic16.jpg}
\caption{万用表的使用}   
\end{figure}
\end{document}
